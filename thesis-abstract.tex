\section*{Abstract}
% \begin{abstract}
    The Boltzmann equation is a fundamental equation in the study of rarefied gas dynamics. While there are vast progresses on the equation on both spatially homogeneous and inhomogeneous cases, there are relatively few results on the case where an external anisotropic shearing force is applied on the gas, despite the importance of such model describing the class of homoenergetic solutions of the equation. Due to the rotational effect introduced from such shearing force, specialized methods are required to analyze the equation, and most existing results are obtained with perturbation method near the isotropic case. There is yet a satisfying theory, mirroring the extensive theory on isotropic Boltzmann equation, to fully analyze the behavior under such anisotropic force.

    In this thesis, we will study the behavior of the solutions of the Boltzmann equation with Maxwellian kernel under such shearing force. In particular, we will investigate the self-similar behavior that emerges from the shear flow, and analyze the long-time behavior of such solution in the Fourier framework. Moreover, we will give a detail description of the behavior of such self-similar profile in the specific case where the gas is under the effect of the Uniform Shear Flow (USF).
% \end{abstract}

\clearpage

\section*{摘要}
% \renewcommand{\abstractname}{摘要}
% \begin{abstract}
    在研究稀疏氣體動力學中,玻爾茲曼方程有著無可比擬的地位。在空間齊性及空間非齊性的經典前提下,對應方程的解的行為均有大量理論結果。然而,同氣體受到非各向同性的剪切力影響時,目前對於相關解的行為卻所知甚少。儘管這種剪切力與均能流解 (homogenergetic solution) 的行為有著密切關係,在剪切力帶來的影響下,已有的手段難以獲得與經典玻爾茲曼方程理論相似的結果。因此,目前已有的結果大多來源於對無剪切力方程的微擾分析。

    在此論文中,我們將在麥克斯韋碰撞核的假定下研究相關玻爾茲曼方程解的行為,並會討論由剪切力所帶來的自相似解結構,及在對應傅立葉空間中分析相關解在大時間尺度下的漸近行為。我們亦會詳細討論由均勻剪切流 (uniform shear flow, USF) 所帶來的影響。
% \end{abstract}

\clearpage